% \iffalse meta-comment
%<*internal>
\iffalse
%</internal>
%<*readme>
----------------------------------------------------------------
penrose --- TikZ library for producing Penrose tilings
E-mail: stacey@math.ntnu.no
Released under the LaTeX Project Public License v1.3c or later
See http://www.latex-project.org/lppl.txt
----------------------------------------------------------------

This package is for the creation of Penrose tilings using either kite-and-dart or rhombuses, and either by manual placement or automatic generation.
%</readme>
%<*internal>
\fi
\def\nameofplainTeX{plain}
\ifx\fmtname\nameofplainTeX\else
  \expandafter\begingroup
\fi
%</internal>
%<*install>
\input docstrip.tex
\keepsilent
\askforoverwritefalse
\preamble
----------------------------------------------------------------
penrose --- TikZ library for producint Penrose tilings
E-mail: stacey@math.ntnu.no
Released under the LaTeX Project Public License v1.3c or later
See http://www.latex-project.org/lppl.txt
----------------------------------------------------------------

\endpreamble
\postamble

Copyright (C) 2014 by Andrew Stacey <stacey@math.ntnu.no>

This work may be distributed and/or modified under the
conditions of the LaTeX Project Public License (LPPL), either
version 1.3c of this license or (at your option) any later
version.  The latest version of this license is in the file:

http://www.latex-project.org/lppl.txt

This work is "maintained" (as per LPPL maintenance status) by
Andrew Stacey.

This work consists of the files  penrose.dtx
                                 penrose_doc.tex
and the derived files            penrose.ins,
                                 penrose.pdf,
                                 tikzlibrarypenrose.code.tex
                                 penrose_doc.pdf
                                 README.txt

\endpostamble
\usedir{tex/latex/penrose}
\generate{
  \file{tikzlibrarypenrose.code.tex}{\from{\jobname.dtx}{library}}
}
%</install>
%<install>\endbatchfile
%<*internal>
\usedir{source/latex/penrose}
\generate{
  \file{\jobname.ins}{\from{\jobname.dtx}{install}}
}
\nopreamble\nopostamble
\usedir{doc/latex/penrose}
\generate{
  \file{README.txt}{\from{\jobname.dtx}{readme}}
}
\ifx\fmtname\nameofplainTeX
  \expandafter\endbatchfile
\else
  \expandafter\endgroup
\fi
%</internal>
%<*driver>
\documentclass[full]{l3doc}
\usepackage[T1]{fontenc}
\usepackage{lmodern}
\usepackage{tikz}
\usepackage{trace}
\usetikzlibrary{penrose}
%\traceoff
%\usepackage[numbered]{hypdoc}
\definecolor{lstbgcolor}{rgb}{0.9,0.9,0.9} 
 
\usepackage{listings}
\lstloadlanguages{[LaTeX]TeX}
\lstset{breakatwhitespace=true,breaklines=true,language=TeX}
 
\usepackage{fancyvrb}

\newenvironment{example}
  {\VerbatimEnvironment
   \begin{VerbatimOut}[gobble=2]{example.out}}
  {\end{VerbatimOut}
   \begin{center}
%   \setlength{\parindent}{0pt}
   \fbox{\begin{minipage}{.9\linewidth}
     \lstset{breakatwhitespace=true,breaklines=true,language=TeX,basicstyle=\small}
     \lstinputlisting[]{example.out}
   \end{minipage}}

   \fbox{\begin{minipage}{.9\linewidth}
     \input{example.out}
   \end{minipage}}
\end{center}
}
\EnableCrossrefs
\CodelineIndex
\RecordChanges
\begin{document}
  \DocInput{\jobname.dtx}
\end{document}
%</driver>
% \fi
%
% \CheckSum{1708}
%
% \CharacterTable
%  {Upper-case    \A\B\C\D\E\F\G\H\I\J\K\L\M\N\O\P\Q\R\S\T\U\V\W\X\Y\Z
%   Lower-case    \a\b\c\d\e\f\g\h\i\j\k\l\m\n\o\p\q\r\s\t\u\v\w\x\y\z
%   Digits        \0\1\2\3\4\5\6\7\8\9
%   Exclamation   \!     Double quote  \"     Hash (number) \#
%   Dollar        \$     Percent       \%     Ampersand     \&
%   Acute accent  \'     Left paren    \(     Right paren   \)
%   Asterisk      \*     Plus          \+     Comma         \,
%   Minus         \-     Point         \.     Solidus       \/
%   Colon         \:     Semicolon     \;     Less than     \<
%   Equals        \=     Greater than  \>     Question mark \?
%   Commercial at \@     Left bracket  \[     Backslash     \\
%   Right bracket \]     Circumflex    \^     Underscore    \_
%   Grave accent  \`     Left brace    \{     Vertical bar  \|
%   Right brace   \}     Tilde         \~}
%
%
% \changes{1.0}{2014/05/07}{Converted to DTX file}
%
% \DoNotIndex{\newcommand,\newenvironment}
%
% \providecommand*{\url}{\texttt}
% \title{The \textsf{Penrose} package}
% \author{Andrew Stacey \\ \url{stacey@math.ntnu.no}}
% \date{1.0 from 2014/05/07}
%
%
% \maketitle
%
% 
% \section{Introduction}
%
% This is a TikZ library for drawing Penrose tiles (both kite/dart and rhombus versions).
% It provides two methods of drawing: one in which an automatic pattern is built, and one where the tiles can be placed ``by hand''.
% The tiles can be shaped and (hopefully!) still fit together.
% For user documentation, see the \Verb+penrose_doc+ file.
%
% \StopEventually{}
%
% \section{Implementation}
%
% \iffalse
%<*library>
% \fi
% \subsection{Initialisation}
%
% We use the \Verb+spath3+ library for manipulating the paths that will make up the tiles.
%
%    \begin{macrocode}
\RequirePackage{spath3}
%    \end{macrocode}
% Now we move in to the realm of \LaTeX3.
%    \begin{macrocode}
\ExplSyntaxOn
%    \end{macrocode}
%
% Start with some basic paths (lines) for the sides of the tiles so that we know that we have well-defined tiles at the outset.
%
%    \begin{macrocode}
\MakeSPath{Penrose path a} {\pgfsyssoftpath@movetotoken{0pt}{0pt}\pgfsyssoftpath@linetotoken{1pt}{0pt}}
\SPathPrepare{Penrose path a}
\CloneSPath {Penrose path a}{Penrose path b}
\CloneSPath {Penrose path a}{Penrose path c}
\CloneSPath {Penrose path a}{Penrose path A}
\CloneSPath {Penrose path a}{Penrose path B}
\CloneSPath {Penrose path a}{Penrose path C}
%    \end{macrocode}
%
% We need a few temporary variables to hold intermediate calculations.
%
%    \begin{macrocode}
\fp_new:N \l__penrose_tmpa_fp
\fp_new:N \l__penrose_tmpb_fp
\fp_new:N \l__penrose_tmpc_fp
\tl_new:N \l__penrose_tmpa_tl
\tl_new:N \l__penrose_tmpb_tl
\tl_new:N \l__penrose_tmpc_tl
%    \end{macrocode}
%
% \subsection{Creating the Tiles}
%
% \begin{macro}{\penrose_normalise_path:n}
% When defining the path for a side, we normalise so that it starts at the origin and ends at \Verb+(1pt,0pt)+.
%    \begin{macrocode}
\cs_new_nopar:Npn \penrose_normalise_path:n #1
{
%    \end{macrocode}
% Get the initial point of the path and convert to floating point.
%    \begin{macrocode}
  \spath_get:nnN {#1} {initial point} \l__penrose_tmpa_tl
  \fp_set:Nn \l__penrose_tmpa_fp {\tl_head:N \l__penrose_tmpa_tl}
  \tl_set:Nx \l__penrose_tmpa_tl {\tl_tail:N \l__penrose_tmpa_tl}
  \fp_set:Nn \l__penrose_tmpb_fp {\tl_head:N \l__penrose_tmpa_tl}
%    \end{macrocode}
% Get the final point of the path, and compute the difference of the final and initial points.
%
% The resulting numbers, say \(a\) and \(b\), will be put into a matrix to rotate and scale the path.
% The formula for the matrix is:
%^^A
% \[
% \frac{1}{a^2 + b^2}
% \begin{bmatrix} a & b \\ -b & a \end{bmatrix}
% \]
%
%    \begin{macrocode}
  \spath_get:nnN {#1} {final point} \l__penrose_tmpa_tl
  \fp_set:Nn \l__penrose_tmpa_fp {\tl_head:N \l__penrose_tmpa_tl - \l__penrose_tmpa_fp}
  \tl_set:Nx \l__penrose_tmpa_tl {\tl_tail:N \l__penrose_tmpa_tl}
  \fp_set:Nn \l__penrose_tmpb_fp {\tl_head:N \l__penrose_tmpa_tl - \l__penrose_tmpb_fp}
%    \end{macrocode}
% Now compute the square of the length of the path for scaling.
%    \begin{macrocode}
  \fp_set:Nn \l__penrose_tmpc_fp {\l__penrose_tmpa_fp^2 + \l__penrose_tmpb_fp^2}
  \fp_set:Nn \l__penrose_tmpa_fp {\l__penrose_tmpa_fp/\l__penrose_tmpc_fp}
  \fp_set:Nn \l__penrose_tmpb_fp {\l__penrose_tmpb_fp/\l__penrose_tmpc_fp}
  \fp_set:Nn \l__penrose_tmpc_fp {-\l__penrose_tmpb_fp}
%    \end{macrocode}
% Now construct the matrix.
%    \begin{macrocode}
  \tl_set:Nx \l__penrose_tmpb_tl { {\fp_use:N \l__penrose_tmpa_fp} {\fp_use:N \l__penrose_tmpb_fp} {\fp_use:N \l__penrose_tmpc_fp} {\fp_use:N \l__penrose_tmpa_fp}}
%    \end{macrocode}
% Get the initial point back again for the translation part.
%    \begin{macrocode}
  \spath_get:nnN {#1} {initial point} \l__penrose_tmpa_tl
%    \end{macrocode}
% But we need to premultiply by the matrix because of how the transformations are applied.
%    \begin{macrocode}
  \fp_set:Nn \l__penrose_tmpa_fp {(-1) * \l__penrose_tmpa_fp * \tl_head:N \l__penrose_tmpa_tl + (-1) * \l__penrose_tmpb_fp * \tl_tail:N \l__penrose_tmpa_tl}
  \fp_set:Nn \l__penrose_tmpb_fp {(-1) * \l__penrose_tmpa_fp * \tl_tail:N \l__penrose_tmpa_tl +  \l__penrose_tmpb_fp * \tl_head:N \l__penrose_tmpa_tl}
%    \end{macrocode}
% Finally, we apply the transformation to the path.
%    \begin{macrocode}
  \tl_put_right:Nx \l__penrose_tmpb_tl {{\fp_to_dim:N \l__penrose_tmpa_fp} {\fp_to_dim:N \l__penrose_tmpb_fp}}
  \spath_transform:nV {#1} \l__penrose_tmpb_tl
}
%    \end{macrocode}
% \end{macro}
%
% \begin{macro}{\SetPenrosePath}
% This sets the path corresponding to a particular side to the current path, and normalises it.
%    \begin{macrocode}
\NewDocumentCommand \SetPenrosePath { m }
{
  \pgfsyssoftpath@getcurrentpath\l__penrose_tmpa_tl
  \spath_clear_new:n {Penrose path #1}
  \spath_put:nnV {Penrose path #1} {path} \l__penrose_tmpa_tl
  \penrose_normalise_path:n {Penrose path #1}
}
%    \end{macrocode}
% \end{macro}
%
% \begin{macro}{\tikz_scan_point:n}
% This is a wrapper around \Verb+\tikz@scan@one@point+ to make it easier to use with \LaTeX3 variables.
%    \begin{macrocode}
\cs_new_nopar:Npn \tikz_scan_point:n #1
{
  \tikz@scan@one@point\pgfutil@firstofone#1\relax
}
\cs_generate_variant:Nn \tikz_scan_point:n {V}
%    \end{macrocode}
% \end{macro}
%
% \begin{macro}{\penrose_make_tile:nnn}
% This builds the tile path from its pieces.
% The arguments are the name of the tile, the descriptions of the sides, and a token list of the coordinates.
%    \begin{macrocode}
\cs_new_nopar:Npn \penrose_make_tile:nnn #1#2#3
{
%    \end{macrocode}
% Get the first coordinate and initialise the path with a move to this point.
%    \begin{macrocode}
  \tl_set:Nn \l__penrose_tmpa_tl {#3}
  \tl_set:Nx \l__penrose_tmpb_tl {\tl_head:N \l__penrose_tmpa_tl}
  \tl_set:Nn \l__penrose_tmpa_tl {\pgfsyssoftpath@movetotoken}
  \tikz_scan_point:V \l__penrose_tmpb_tl
  \tl_put_right:Nx \l__penrose_tmpa_tl {{\dim_use:N \pgf@x}{\dim_use:N \pgf@y}}
  \spath_clear_new:n {Penrose path tile #1}
  \spath_put:nnV {Penrose path tile #1} {path} \l__penrose_tmpa_tl
%    \end{macrocode}
% Now we have our path initialised, we can start appending the side paths according to the specification in the second argument.
%
% We append the initial coordinate to the end of the list to make a closed cycle.
%    \begin{macrocode}
  \tl_set:Nn \l__penrose_tmpa_tl {#3}
  \tl_put_right:Nx \l__penrose_tmpa_tl {{\tl_head:N \l__penrose_tmpa_tl}}
%    \end{macrocode}
% Now we walk through the description of the sides, adding the specified paths to our tile path.
%    \begin{macrocode}
  \tl_map_inline:nn {#2} {
%    \end{macrocode}
% Clone the path for this side.
%    \begin{macrocode}
    \spath_clone:nn {Penrose path ##1} {Penrose path tmpa}
%    \end{macrocode}
% Strip off the next coordinate, and convert it to a point.
%    \begin{macrocode}
    \tl_set:Nx \l__penrose_tmpb_tl {\tl_head:N \l__penrose_tmpa_tl}
    \tl_set:Nx \l__penrose_tmpa_tl {\tl_tail:N \l__penrose_tmpa_tl}
    \tikz_scan_point:V \l__penrose_tmpb_tl
%    \end{macrocode}
% Store the resulting coordinate.
%    \begin{macrocode}
    \fp_set:Nn \l__penrose_tmpa_fp { \pgf@x }
    \fp_set:Nn \l__penrose_tmpb_fp { \pgf@y }
%    \end{macrocode}
% Now get the next coordinate.
%    \begin{macrocode}
    \tl_set:Nx \l__penrose_tmpb_tl {\tl_head:N \l__penrose_tmpa_tl}
    \tikz_scan_point:V \l__penrose_tmpb_tl
%    \end{macrocode}
% We want the difference between the two coordinates.
%    \begin{macrocode}
    \fp_set:Nn \l__penrose_tmpa_fp {\pgf@x - \l__penrose_tmpa_fp}
    \fp_set:Nn \l__penrose_tmpb_fp {\pgf@y - \l__penrose_tmpb_fp}
%    \end{macrocode}
% This is converted into a transformation matrix.
%    \begin{macrocode}
    \fp_set:Nn \l__penrose_tmpc_fp {-\l__penrose_tmpb_fp}
    \tl_set:Nx \l__penrose_tmpb_tl {{\fp_use:N \l__penrose_tmpa_fp} {\fp_use:N \l__penrose_tmpc_fp} {\fp_use:N \l__penrose_tmpb_fp} {\fp_use:N \l__penrose_tmpa_fp} {0} {0}}
%    \end{macrocode}
% The transformation is applied to the cloned path.
%    \begin{macrocode}
    \spath_transform:nV {Penrose path tmpa} \l__penrose_tmpb_tl
%    \end{macrocode}
% And this is welded to the tile path.
%    \begin{macrocode}
    \spath_weld:nn {Penrose path tile #1} {Penrose path tmpa}
  }
%    \end{macrocode}
% At the end we close the path.
%    \begin{macrocode}
  \spath_close_path:n {Penrose path tile #1}
}
%    \end{macrocode}
% \end{macro}
%
% \begin{macro}{\penrose_make_tile:nn}
% A wrapper around the above which allows us to specify the second two arguments as two items in a token list.
%    \begin{macrocode}
\cs_new_nopar:Npn \penrose_make_tile:nn #1#2
{
  \penrose_make_tile:nnn {#1} #2
}
\cs_generate_variant:Nn \penrose_make_tile:nn {nV}
%    \end{macrocode}
% \end{macro}
%
% \subsection{Specifying the Tiles}
%
% The tile specifications are contained in a \Verb+prop+.
%    \begin{macrocode}
\prop_new:N \g__penrose_tiles_prop
%    \end{macrocode}
%
% \begin{macro}{\tl_add_coordinate:Nnn}
% Process a coordinate through \Verb+fp+ and adds it to a token list.
%    \begin{macrocode}
\cs_new_nopar:Npn \tl_add_coordinate:Nnn #1#2#3 {
  \fp_set:Nn \l__penrose_tmpa_fp{#2}
  \fp_set:Nn \l__penrose_tmpb_fp{#3}
  \tl_put_right:Nx #1 {{(\fp_use:N \l__penrose_tmpa_fp, \fp_use:N \l__penrose_tmpb_fp)}}
}
%    \end{macrocode}
% \end{macro}
%
% Now we specify the tiles.
% The specification is a clockwise list of the vertices together with the labels of the corresponding sides.
% There are three basic paths, \Verb+a+, \Verb+b+, \Verb+c+, and their complements (which are capitalised).
%
% \begin{itemize}
% \item Thin Rhombus.
%    \begin{macrocode}
\tl_clear:N \l__penrose_tmpa_tl
\tl_add_coordinate:Nnn \l__penrose_tmpa_tl {0}{0}
\tl_add_coordinate:Nnn \l__penrose_tmpa_tl {cosd(18)}{sind(18)}
\tl_add_coordinate:Nnn \l__penrose_tmpa_tl {2*cosd(18)}{0}
\tl_add_coordinate:Nnn \l__penrose_tmpa_tl {cosd(18)}{-sind(18)}

\prop_gput:Nnx \g__penrose_tiles_prop {thin~ rhombus}  {{a A B b} {\tl_use:N \l__penrose_tmpa_tl}}
%    \end{macrocode}
%
% \item Thick Rhombus.
%    \begin{macrocode}
\tl_clear:N \l__penrose_tmpa_tl
\tl_add_coordinate:Nnn \l__penrose_tmpa_tl {0}{0}
\tl_add_coordinate:Nnn \l__penrose_tmpa_tl {cosd(36)}{sind(36)}
\tl_add_coordinate:Nnn \l__penrose_tmpa_tl {2*cosd(36)}{0}
\tl_add_coordinate:Nnn \l__penrose_tmpa_tl {cosd(36)}{-sind(36)}

\prop_gput:Nnx \g__penrose_tiles_prop {thick~ rhombus}  {{B a A b} {\tl_use:N \l__penrose_tmpa_tl}}
%    \end{macrocode}
%
% \item Dart.
%    \begin{macrocode}
\tl_clear:N \l__penrose_tmpa_tl
\tl_add_coordinate:Nnn \l__penrose_tmpa_tl {0}{0}
\tl_add_coordinate:Nnn \l__penrose_tmpa_tl {2*sind(18)*cosd(108)}{2*sind(18)*sind(108)}
\tl_add_coordinate:Nnn \l__penrose_tmpa_tl {2*sind(18)}{0}
\tl_add_coordinate:Nnn \l__penrose_tmpa_tl {2*sind(18)*cosd(108)}{-2*sind(18)*sind(108)}

\prop_gput:Nnx \g__penrose_tiles_prop {dart}  {{c a A C} {\tl_use:N \l__penrose_tmpa_tl}}
%    \end{macrocode}
%
% \item Kite.
%    \begin{macrocode}
\tl_clear:N \l__penrose_tmpa_tl
\tl_add_coordinate:Nnn \l__penrose_tmpa_tl {0}{0}
\tl_add_coordinate:Nnn \l__penrose_tmpa_tl {cosd(36)}{sind(36)}
\tl_add_coordinate:Nnn \l__penrose_tmpa_tl {1}{0}
\tl_add_coordinate:Nnn \l__penrose_tmpa_tl {cosd(36)}{-sind(36)}

\prop_gput:Nnx \g__penrose_tiles_prop {kite}  {{a c C A} {\tl_use:N \l__penrose_tmpa_tl}}
%    \end{macrocode}
%
% \item Golden Triangle.
%    \begin{macrocode}
\tl_clear:N \l__penrose_tmpa_tl
\tl_add_coordinate:Nnn \l__penrose_tmpa_tl {0}{0}
\tl_add_coordinate:Nnn \l__penrose_tmpa_tl {cosd(18)}{sind(18)}
\tl_add_coordinate:Nnn \l__penrose_tmpa_tl {cosd(18)}{-sind(18)}

\prop_gput:Nnx \g__penrose_tiles_prop {golden~ triangle}  {{a c b} {\tl_use:N \l__penrose_tmpa_tl}}
%    \end{macrocode}
%
% \item Reverse Golden Triangle.
%    \begin{macrocode}
\tl_clear:N \l__penrose_tmpa_tl
\tl_add_coordinate:Nnn \l__penrose_tmpa_tl {0}{0}
\tl_add_coordinate:Nnn \l__penrose_tmpa_tl {cosd(18)}{sind(18)}
\tl_add_coordinate:Nnn \l__penrose_tmpa_tl {cosd(18)}{-sind(18)}

\prop_gput:Nnx \g__penrose_tiles_prop {reverse~ golden~ triangle}  {{B C A} {\tl_use:N \l__penrose_tmpa_tl}}
%    \end{macrocode}
%
% \item Golden Gnomon
%    \begin{macrocode}
\tl_clear:N \l__penrose_tmpa_tl
\tl_add_coordinate:Nnn \l__penrose_tmpa_tl {0}{0}
\tl_add_coordinate:Nnn \l__penrose_tmpa_tl {cosd(36)}{sind(36)}
\tl_add_coordinate:Nnn \l__penrose_tmpa_tl {2*cosd(36)}{0}

\prop_gput:Nnx \g__penrose_tiles_prop {golden~ gnomon}  {{C b A} {\tl_use:N \l__penrose_tmpa_tl}}
%    \end{macrocode}
%
% \item Reverse Golden Gnomon
%    \begin{macrocode}
\tl_clear:N \l__penrose_tmpa_tl
\tl_add_coordinate:Nnn \l__penrose_tmpa_tl {0}{0}
\tl_add_coordinate:Nnn \l__penrose_tmpa_tl {2*cosd(36)}{0}
\tl_add_coordinate:Nnn \l__penrose_tmpa_tl {cosd(36)}{-sind(36)}
\prop_gput:Nnx \g__penrose_tiles_prop {reverse~ golden~ gnomon}  {{a B c} {\tl_use:N \l__penrose_tmpa_tl}}
%    \end{macrocode}
% \end{itemize}
%
% \begin{macro}{\MakePenroseTile}
% This is the user wrapper around the tile creation macros.
%    \begin{macrocode}
\NewDocumentCommand \MakePenroseTile {m}
{
  \prop_get:NnN \g__penrose_tiles_prop {#1} \l__penrose_tmpa_tl
  \penrose_make_tile:nV {#1} \l__penrose_tmpa_tl
}
%    \end{macrocode}
% \end{macro}
%
% \begin{macro}{\UsePenroseTile}
% This is the command that actually places a tile on the page.
% The first argument is optional and is for styling.
%    \begin{macrocode}
\NewDocumentCommand \UsePenroseTile {O{} m} 
{
%    \end{macrocode}
% We need to transform the tile to correspond to the current transformation matrix.
% To ensure that we only transform the current tile, we clone it first.
%    \begin{macrocode}
  \spath_clone:nn {Penrose path tile #2} {Penrose path tmpa}
%    \end{macrocode}
% The transformation matrix returned by PGF appears to be transposed from what it should be.
% (This needs a little more investigation, it might be that I've implemented the multiplication incorrectly here.)
%    \begin{macrocode}
  \pgfgettransform \l__penrose_tmpa_tl
  \tl_clear:N \l__penrose_tmpb_tl
  \tl_set:Nx \l__penrose_tmpb_tl {{\tl_head:N \l__penrose_tmpa_tl}}
  \tl_set:Nx \l__penrose_tmpa_tl {\tl_tail:N \l__penrose_tmpa_tl}
  \tl_put_right:Nx \l__penrose_tmpb_tl {{\tl_item:Nn \l__penrose_tmpa_tl {2}}}
  \tl_put_right:Nx \l__penrose_tmpb_tl {{\tl_item:Nn \l__penrose_tmpa_tl {1}}}
  \tl_set:Nx \l__penrose_tmpa_tl {\tl_tail:N \l__penrose_tmpa_tl}
  \tl_set:Nx \l__penrose_tmpa_tl {\tl_tail:N \l__penrose_tmpa_tl}
  \tl_put_right:NV \l__penrose_tmpb_tl \l__penrose_tmpa_tl
%    \end{macrocode}
% Apply the transformation, protocol the path, and render it.
%    \begin{macrocode}
  \spath_transform:nV {Penrose path tmpa} \l__penrose_tmpb_tl
  \spath_protocol_path:n {Penrose path tmpa}
  \spath_tikz_path:nn {#1}{Penrose path tmpa}
}
%    \end{macrocode}
% \end{macro}
%
% This is a style for a user to take a path and make it into the path for one of the sides.
% It needs to store both that side and the reverse.
%    \begin{macrocode}
\tikzset{
  save~ Penrose~ path/.code={
    \tikz@addmode{
%    \end{macrocode}
% Get the current path.
%    \begin{macrocode}
      \pgfsyssoftpath@getcurrentpath\l__penrose_tmpa_tl
%    \end{macrocode}
% Clear the receiving path, and store the current path in it.
%    \begin{macrocode}
      \spath_clear_new:n {Penrose path #1}
      \spath_put:nnV {Penrose path #1} {path} \l__penrose_tmpa_tl
%    \end{macrocode}
% Normalise the path.
%    \begin{macrocode}
      \penrose_normalise_path:n {Penrose path #1}
%    \end{macrocode}
% Now create the reverse path.
% The name is the upper case version.
%    \begin{macrocode}
      \tl_to_uppercase:n {\tl_set:Nx \l__penrose_tmpa_tl {#1}}
%    \end{macrocode}
% Clone the path.
%    \begin{macrocode}
      \spath_clone:nn {Penrose path #1} {Penrose path \tl_use:N \l__penrose_tmpa_tl }
%    \end{macrocode}
% Reverse it.
%    \begin{macrocode}
      \spath_reverse:n {Penrose path \tl_use:N \l__penrose_tmpa_tl}
%    \end{macrocode}
% Swap the start and end.
%    \begin{macrocode}
      \spath_transform:nnnnnnn {Penrose path \tl_use:N \l__penrose_tmpa_tl} {-1} {0} {0} {-1} {1} {0}
    }
  },
  expand~ key/.code={
    \exp_args:NV \pgfkeysalso #1
  }
}
%    \end{macrocode}
%
% Create the basic tile shapes.
%    \begin{macrocode}
\MakePenroseTile {thin~ rhombus}
\MakePenroseTile {thick~ rhombus} 
\MakePenroseTile {dart}
\MakePenroseTile {kite}
\MakePenroseTile {golden~ triangle}
\MakePenroseTile {reverse~ golden~ triangle}
\MakePenroseTile {golden~ gnomon}
\MakePenroseTile {reverse~ golden~ gnomon}
%    \end{macrocode}
%
% \subsection{Lindenmayer System}
% 
% This is an implementation of the Lindenmayer System description of Penrose tilings as a way of generating tilings from a specific starting seed.
%
% The implementation uses \Verb+prop+s to store \emph{rules} and \emph{actions}.
% The rules are used to expand the starting seed to a certain level, after which the actions are carried out.
% The syntax is based on the PGF library, but as we're already using \LaTeX3 it is reimplemented in that.
%
% These are the rules for generating rhombus tilings.
%    \begin{macrocode}
\prop_new:N \g__penrose_rhombus_lms_rule_prop
\prop_put:Nnn \g__penrose_rhombus_lms_rule_prop {T} {[f*sT][f>g]}
\prop_put:Nnn \g__penrose_rhombus_lms_rule_prop {t} {[f_st][f>G]}
\prop_put:Nnn \g__penrose_rhombus_lms_rule_prop {G} {[f+sG][sf>g][sf*sT]}
\prop_put:Nnn \g__penrose_rhombus_lms_rule_prop {g} {[f-sg][sf>G][sf_st]}
%    \end{macrocode}
%
% These are the rules for generating kite and dart tilings.
%    \begin{macrocode}
\prop_new:N \g__penrose_kite_lms_rule_prop
\prop_put:Nnn \g__penrose_kite_lms_rule_prop {T} {[f*sT][f>st][+sg]}
\prop_put:Nnn \g__penrose_kite_lms_rule_prop {t} {[f_st][f>sT][-sG]}
\prop_put:Nnn \g__penrose_kite_lms_rule_prop {G} {[f*+sG][sT]}
\prop_put:Nnn \g__penrose_kite_lms_rule_prop {g} {[f-_sg][st]}
%    \end{macrocode}
%
% Each of the standard tilings can also be drawn using triangles using the same rules.
%    \begin{macrocode}
\prop_set_eq:NN \g__penrose_rtriangle_lms_rule_prop \g__penrose_rhombus_lms_rule_prop
\prop_set_eq:NN \g__penrose_ktriangle_lms_rule_prop \g__penrose_kite_lms_rule_prop
%    \end{macrocode}
%
% These hold the various actions.
%    \begin{macrocode}
\prop_new:N \g__penrose_default_lms_action_prop
\prop_new:N \g__penrose_rhombus_lms_action_prop
\prop_new:N \g__penrose_kite_lms_action_prop
\prop_new:N \g__penrose_rtriangle_lms_action_prop
\prop_new:N \g__penrose_ktriangle_lms_action_prop
%    \end{macrocode}
%
% We need some parameters.
%    \begin{macrocode}
\dim_new:N \l__penrose_step_dim
\dim_set:Nn \l__penrose_step_dim {1cm}
%    \end{macrocode}
%
% These are the defaults, which will be used in all the rule sets.
%    \begin{macrocode}
\prop_put:Nnn \g__penrose_default_lms_action_prop {[} {\group_begin:}
\prop_put:Nnn \g__penrose_default_lms_action_prop {]} {\group_end:}
\prop_put:Nnn \g__penrose_default_lms_action_prop {f} {\pgftransformxshift{\l__penrose_step_dim}}
\prop_put:Nnn \g__penrose_default_lms_action_prop {s} {
  \fp_set:Nn \l__penrose_tmpa_fp { 2 * sind(18) * \l__penrose_step_dim }
  \dim_set:Nn \l__penrose_step_dim {\fp_to_dim:N \l__penrose_tmpa_fp}
}
%    \end{macrocode}
%
% The rhombus rules need a variety of turns.
%    \begin{macrocode}
\prop_put:Nnn \g__penrose_rhombus_lms_action_prop {+} {\pgftransformrotate{144}}
\prop_put:Nnn \g__penrose_rhombus_lms_action_prop {*} {\pgftransformrotate{108}}
\prop_put:Nnn \g__penrose_rhombus_lms_action_prop {-} {\pgftransformrotate{216}}
\prop_put:Nnn \g__penrose_rhombus_lms_action_prop {_} {\pgftransformrotate{252}}
\prop_put:Nnn \g__penrose_rhombus_lms_action_prop {>} {\pgftransformrotate{180}}
%    \end{macrocode}
%
% Up to now, the actions for the rhombus and its triangle replacement are the same.
%    \begin{macrocode}
\prop_set_eq:NN \g__penrose_rtriangle_lms_action_prop  \g__penrose_rhombus_lms_action_prop
%    \end{macrocode}
%
% Now we do the actions that actually draw something.
%    \begin{macrocode}
\prop_put:Nnn \g__penrose_rhombus_lms_action_prop {T} {
  \group_begin:
%    \end{macrocode}
% As we go through, we keep track of how many tiles we've drawn.
%    \begin{macrocode}
  \int_gincr:N \l__penrose_tile_int
%    \end{macrocode}
% Set up the position, size, and angle correctly.
%    \begin{macrocode}
  \pgftransformrotate{198}
  \fp_set:Nn \l__penrose_tmpa_fp {\l__penrose_step_dim*2*cosd(18)}
  \pgftransformxshift{-\fp_to_dim:N \l__penrose_tmpa_fp}
  \fp_set:Nn \l__penrose_tmpa_fp {\l__penrose_step_dim/1cm}
  \pgftransformscale{\fp_use:N \l__penrose_tmpa_fp}
%    \end{macrocode}
% Now we draw the thin rhombus, applying every style we can possibly imagine.
% The \Verb+Penrose tile+ style gets the current tile and total tile numbers passed to it.
%    \begin{macrocode}
\tl_set:Nx \l__penrose_tmpc_tl {{\int_use:N  \l__penrose_tile_int} { \int_use:N \l__penrose_tiles_int}}
  \UsePenroseTile[every~ Penrose~ tile/.try, every~ thin~ rhombus/.try, Penrose~ tile~ \int_use:N \l__penrose_tile_int/.try, Penrose~ tile/.try/.expand~ once=\l__penrose_tmpc_tl ]{thin~rhombus}
  \group_end:
}
%    \end{macrocode}
%
% Same for the thick rhombus.
%    \begin{macrocode}
\prop_put:Nnn \g__penrose_rhombus_lms_action_prop {G} {
  \group_begin:
  \int_gincr:N \l__penrose_tile_int
  \fp_set:Nn \l__penrose_tmpa_fp {\l__penrose_step_dim/1cm/(2*cosd(36))}
  \pgftransformscale{\fp_use:N \l__penrose_tmpa_fp}
  \tl_set:Nx \l__penrose_tmpc_tl {{\int_use:N  \l__penrose_tile_int} { \int_use:N \l__penrose_tiles_int}}
\UsePenroseTile[every~ Penrose~ tile/.try, every~ thick~ rhombus/.try, Penrose~ tile~ \int_use:N \l__penrose_tile_int/.try, Penrose~ tile/.try/.expand~ once=\l__penrose_tmpc_tl ]{thick~rhombus}
  \group_end:
}
%    \end{macrocode}
%
% Now we do the same for the kite and dart tiling.
%    \begin{macrocode}
\prop_put:Nnn \g__penrose_kite_lms_action_prop {+} {\pgftransformrotate{36}}
\prop_put:Nnn \g__penrose_kite_lms_action_prop {*} {\pgftransformrotate{108}}
\prop_put:Nnn \g__penrose_kite_lms_action_prop {-} {\pgftransformrotate{-36}}
\prop_put:Nnn \g__penrose_kite_lms_action_prop {_} {\pgftransformrotate{-108}}
\prop_put:Nnn \g__penrose_kite_lms_action_prop {>} {\pgftransformrotate{180}}
%    \end{macrocode}
%
%    \begin{macrocode}
\prop_set_eq:NN \g__penrose_ktriangle_lms_action_prop  \g__penrose_kite_lms_action_prop
%    \end{macrocode}
%
%    \begin{macrocode}
\prop_put:Nnn \g__penrose_kite_lms_action_prop {T} {
  \group_begin:
  \int_gincr:N \l__penrose_tile_int
  \pgftransformrotate{36}
  \fp_set:Nn \l__penrose_tmpa_fp {\l__penrose_step_dim/1cm}
  \pgftransformscale{\fp_use:N \l__penrose_tmpa_fp}
  \tl_set:Nx \l__penrose_tmpc_tl {{\int_use:N  \l__penrose_tile_int} { \int_use:N \l__penrose_tiles_int}}
  \UsePenroseTile[every~ Penrose~ tile/.try, every~ kite/.try, Penrose~ tile~ \int_use:N \l__penrose_tile_int/.try, Penrose~ tile/.try/.expand~ once=\l__penrose_tmpc_tl ]{kite}
  \group_end:
}
%    \end{macrocode}
%
%    \begin{macrocode}
\prop_put:Nnn \g__penrose_kite_lms_action_prop {g} {
  \group_begin:
  \int_gincr:N \l__penrose_tile_int
  \pgftransformrotate{144}
  \pgftransformxshift{-\l__penrose_step_dim * 2 * sin(18)}
  \fp_set:Nn \l__penrose_tmpa_fp {\l__penrose_step_dim/1cm}
  \pgftransformscale{\fp_use:N \l__penrose_tmpa_fp}
  \tl_set:Nx \l__penrose_tmpc_tl {{\int_use:N  \l__penrose_tile_int} { \int_use:N \l__penrose_tiles_int}}
  \UsePenroseTile[every~ Penrose~ tile/.try, every~ dart/.try, Penrose~ tile~ \int_use:N \l__penrose_tile_int/.try, Penrose~ tile/.try/.expand~ once=\l__penrose_tmpc_tl ]{dart}
  \group_end:
}
%    \end{macrocode}
%
% Now we set up the actions for the triangle variations.
%    \begin{macrocode}
\prop_put:Nnn \g__penrose_rtriangle_lms_action_prop {T} {
  \group_begin:
  \int_gincr:N \l__penrose_tile_int
  \pgftransformrotate{18}
  \fp_set:Nn \l__penrose_tmpa_fp {\l__penrose_step_dim/1cm}
  \pgftransformscale{\fp_use:N \l__penrose_tmpa_fp}
  \tl_set:Nx \l__penrose_tmpc_tl {{\int_use:N  \l__penrose_tile_int} { \int_use:N \l__penrose_tiles_int}}
  \UsePenroseTile[every~ Penrose~ tile/.try, every~ reverse~ golden~ triangle/.try, Penrose~ tile~ \int_use:N \l__penrose_tile_int/.try, Penrose~ tile/.try/.expand~ once=\l__penrose_tmpc_tl ]{reverse~ golden~ triangle}
  \group_end:
}
%    \end{macrocode}
%
%    \begin{macrocode}
\prop_put:Nnn \g__penrose_rtriangle_lms_action_prop {t} {
  \group_begin:
  \int_gincr:N \l__penrose_tile_int
  \pgftransformrotate{-18}
  \fp_set:Nn \l__penrose_tmpa_fp {\l__penrose_step_dim/1cm}
  \pgftransformscale{\fp_use:N \l__penrose_tmpa_fp}
  \tl_set:Nx \l__penrose_tmpc_tl {{\int_use:N  \l__penrose_tile_int} { \int_use:N \l__penrose_tiles_int}}
  \tl_set:Nx \l__penrose_tmpc_tl {{\int_use:N  \l__penrose_tile_int} { \int_use:N \l__penrose_tiles_int}}
  \UsePenroseTile[every~ Penrose~ tile/.try, every~ golden~ triangle/.try, Penrose~ tile~ \int_use:N \l__penrose_tile_int/.try, Penrose~ tile/.try/.expand~ once=\l__penrose_tmpc_tl ]{golden~ triangle}
  \group_end:
}
%    \end{macrocode}
%
%    \begin{macrocode}
\prop_put:Nnn \g__penrose_rtriangle_lms_action_prop {G} {
  \group_begin:
  \int_gincr:N \l__penrose_tile_int
  \pgftransformrotate{180}
  \pgftransformxshift{-\l__penrose_step_dim}
  \fp_set:Nn \l__penrose_tmpa_fp {\l__penrose_step_dim/1cm/(2*cosd(36))}
  \pgftransformscale{\fp_use:N \l__penrose_tmpa_fp}
  \tl_set:Nx \l__penrose_tmpc_tl {{\int_use:N  \l__penrose_tile_int} { \int_use:N \l__penrose_tiles_int}}
  \UsePenroseTile[every~ Penrose~ tile/.try, every~ reverse~ golden~ gnomon/.try, Penrose~ tile~ \int_use:N \l__penrose_tile_int/.try, Penrose~ tile/.try/.expand~ once=\l__penrose_tmpc_tl ]{reverse~ golden~ gnomon}
  \group_end:
}
%    \end{macrocode}
%
%    \begin{macrocode}
\prop_put:Nnn \g__penrose_rtriangle_lms_action_prop {g} {
  \group_begin:
  \int_gincr:N \l__penrose_tile_int
  \pgftransformrotate{180}
  \pgftransformxshift{-\l__penrose_step_dim}
  \fp_set:Nn \l__penrose_tmpa_fp {\l__penrose_step_dim/1cm/(2*cosd(36))}
  \pgftransformscale{\fp_use:N \l__penrose_tmpa_fp}
  \tl_set:Nx \l__penrose_tmpc_tl {{\int_use:N  \l__penrose_tile_int} { \int_use:N \l__penrose_tiles_int}}
  \UsePenroseTile[every~ Penrose~ tile/.try, every~ golden~ gnomon/.try, Penrose~ tile~ \int_use:N \l__penrose_tile_int/.try, Penrose~ tile/.try/.expand~ once=\l__penrose_tmpc_tl ]{golden~ gnomon}
  \group_end:
}
%    \end{macrocode}
%
%    \begin{macrocode}
\prop_put:Nnn \g__penrose_ktriangle_lms_action_prop {T} {
  \group_begin:
  \int_gincr:N \l__penrose_tile_int
  \pgftransformrotate{18}
  \fp_set:Nn \l__penrose_tmpa_fp {\l__penrose_step_dim/1cm}
  \pgftransformscale{\fp_use:N \l__penrose_tmpa_fp}
  \tl_set:Nx \l__penrose_tmpc_tl {{\int_use:N  \l__penrose_tile_int} { \int_use:N \l__penrose_tiles_int}}
  \UsePenroseTile[every~ Penrose~ tile/.try, every~ reverse~ golden~ triangle/.try, Penrose~ tile~ \int_use:N \l__penrose_tile_int/.try, Penrose~ tile/.try/.expand~ once=\l__penrose_tmpc_tl ]{reverse~ golden~ triangle}
  \group_end:
}
%    \end{macrocode}
%
%    \begin{macrocode}
\prop_put:Nnn \g__penrose_ktriangle_lms_action_prop {t} {
  \group_begin:
  \int_gincr:N \l__penrose_tile_int
  \pgftransformrotate{-18}
  \fp_set:Nn \l__penrose_tmpa_fp {\l__penrose_step_dim/1cm}
  \pgftransformscale{\fp_use:N \l__penrose_tmpa_fp}
  \tl_set:Nx \l__penrose_tmpc_tl {{\int_use:N  \l__penrose_tile_int} { \int_use:N \l__penrose_tiles_int}}
  \UsePenroseTile[every~ Penrose~ tile/.try, every~ golden~ triangle/.try, Penrose~ tile~ \int_use:N \l__penrose_tile_int/.try, Penrose~ tile/.try/.expand~ once=\l__penrose_tmpc_tl ]{golden~ triangle}
  \group_end:
}
%    \end{macrocode}
%
%    \begin{macrocode}
\prop_put:Nnn \g__penrose_ktriangle_lms_action_prop {G} {
  \group_begin:
  \int_gincr:N \l__penrose_tile_int
  \pgftransformrotate{180}
  \pgftransformxshift{-\l__penrose_step_dim}
  \fp_set:Nn \l__penrose_tmpa_fp {\l__penrose_step_dim/1cm/(2*cosd(36))}
  \pgftransformscale{\fp_use:N \l__penrose_tmpa_fp}
  \tl_set:Nx \l__penrose_tmpc_tl {{\int_use:N  \l__penrose_tile_int} { \int_use:N \l__penrose_tiles_int}}
  \UsePenroseTile[every~ Penrose~ tile/.try, every~ reverse~ golden~ gnomon/.try, Penrose~ tile~ \int_use:N \l__penrose_tile_int/.try, Penrose~ tile/.try/.expand~ once=\l__penrose_tmpc_tl ]{reverse~ golden~ gnomon}
  \group_end:
}
%    \end{macrocode}
%
%    \begin{macrocode}
\prop_put:Nnn \g__penrose_ktriangle_lms_action_prop {g} {
  \group_begin:
  \int_gincr:N \l__penrose_tile_int
  \pgftransformrotate{180}
  \pgftransformxshift{-\l__penrose_step_dim}
  \fp_set:Nn \l__penrose_tmpa_fp {\l__penrose_step_dim/1cm/(2*cosd(36))}
  \pgftransformscale{\fp_use:N \l__penrose_tmpa_fp}
  \tl_set:Nx \l__penrose_tmpc_tl {{\int_use:N  \l__penrose_tile_int} { \int_use:N \l__penrose_tiles_int}}
  \UsePenroseTile[every~ Penrose~ tile/.try, every~ golden~ gnomon/.try, Penrose~ tile~ \int_use:N \l__penrose_tile_int/.try, Penrose~ tile/.try/.expand~ once=\l__penrose_tmpc_tl ]{golden~ gnomon}
  \group_end:
}
%    \end{macrocode}
%
% \begin{macro}{\penrose_make_lms:Nnnn}
% This creates the token list of actions, starting with the seed.
% The arguments are: a token list to store the result in, the name of the system, the number of iterations, and the initial state.
%    \begin{macrocode}
\cs_new_nopar:Npn \penrose_make_lms:Nnnn #1#2#3#4
{
  \group_begin:
%    \end{macrocode}
% On the first time round, we start with the given seed.
%    \begin{macrocode}
  \tl_set:Nn \l__penrose_tmpb_tl {#4}
%    \end{macrocode}
% We repeat the specified number of times.
%    \begin{macrocode}
  \prg_replicate:nn {#3} {
%    \end{macrocode}
% Duplicate the current state.
%    \begin{macrocode}
    \tl_set_eq:NN \l__penrose_tmpa_tl \l__penrose_tmpb_tl
%    \end{macrocode}
% Clear the receiving token list.
%    \begin{macrocode}
    \tl_clear:N \l__penrose_tmpb_tl
%    \end{macrocode}
% Walk through the current list, appending to the receiving list according to the rules.
%    \begin{macrocode}
    \tl_map_inline:Nn \l__penrose_tmpa_tl
    {
%    \end{macrocode}
% If a rule exists, copy that.
%    \begin{macrocode}
      \prop_if_in:cnTF {g__penrose_#2_lms_rule_prop} {##1}
      {
        \tl_put_right:Nx \l__penrose_tmpb_tl {\prop_get:cn {g__penrose_#2_lms_rule_prop} {##1} }
      }
      {
%    \end{macrocode}
% Otherwise, just copy the token.
%    \begin{macrocode}
        \tl_put_right:Nn \l__penrose_tmpb_tl {##1}
      }
    }
  }
%    \end{macrocode}
% We've done all this inside a group, now pass the result outside.
%    \begin{macrocode}
  \tl_set:Nn \l__penrose_tmpa_tl {
    \group_end:
    \tl_set:Nn #1
  }
  \tl_put_right:Nx \l__penrose_tmpa_tl {{\tl_use:N \l__penrose_tmpb_tl}}
  \tl_use:N \l__penrose_tmpa_tl
}
\cs_generate_variant:Nn \penrose_make_lms:Nnnn {Nnnx}
%    \end{macrocode}
% \end{macro}
%
% \begin{macro}{\penrose_invoke_lms:Nn}
% This carries out the actions specified by the resulting rules.
%    \begin{macrocode}
\cs_new_nopar:Npn \penrose_invoke_lms:Nn #1#2
{
  \group_begin:
%    \end{macrocode}
% Walk through the given list, carrying out the corresponding action if it exists.
% If not, look at the default.
% Otherwise, just do nothing.
%    \begin{macrocode}
  \tl_map_inline:Nn #1 {
    \prop_if_in:cnTF {g__penrose_#2_lms_action_prop} {##1}
    {
      \prop_get:cn {g__penrose_#2_lms_action_prop} {##1}
    }
    {
      \prop_if_in:cnT {g__penrose_default_lms_action_prop} {##1}
      {
        \prop_get:cn {g__penrose_default_lms_action_prop} {##1}
      }
    }
  }
  \group_end:
}
%    \end{macrocode}
% \end{macro}
%
% We keep track of the number of tiles.
%    \begin{macrocode}
\int_new:N \l__penrose_tile_int
\int_new:N \l__penrose_tiles_int
%    \end{macrocode}
%
% \begin{macro}{\PenroseDecomposition}
% This is the user macro to invoke the decomposition.
% The arguments are: optional styles, the name, number of iterations, and starting seed.
%    \begin{macrocode}
\NewDocumentCommand \PenroseDecomposition { O{} m m m }
{
  \group_begin:
  \tikzset{#1}
  \penrose_make_lms:Nnnx \l__penrose_tmpa_tl {#2} {#3} {#4}
  \penrose_count_lms:N \l__penrose_tmpa_tl
  \int_gzero:N \l__penrose_tile_int
  \penrose_invoke_lms:Nn \l__penrose_tmpa_tl {#2}
  \group_end:
}
%    \end{macrocode}
% \end{macro}
%
% \begin{macro}{\penrose_count_lms:N}
% This counts the number of tiles in the string.
%    \begin{macrocode}
\cs_new_nopar:Npn \penrose_count_lms:N #1
{
  \int_gzero:N \l__penrose_tiles_int
  \tl_map_inline:Nn #1
  {
    \tl_if_eq:nnT {##1} {T}
    {
      \int_incr:N \l__penrose_tiles_int
    }
    \tl_if_eq:nnT {##1} {t}
    {
      \int_incr:N \l__penrose_tiles_int
    }
    \tl_if_eq:nnT {##1} {G}
    {
      \int_incr:N \l__penrose_tiles_int
    }
    \tl_if_eq:nnT {##1} {g}
    {
      \int_incr:N \l__penrose_tiles_int
    }
  }
}
%    \end{macrocode}
% \end{macro}
%
% This is a \Verb+\tikzset+ mechanism for setting the dimensions of the tiling.
%    \begin{macrocode}
\tikzset{
  Penrose~ step/.code={
    \dim_set:Nn \l__penrose_step_dim {#1}
  }
}
%    \end{macrocode}
%
% We're done with \LaTeX3, so turn off the syntax.
%    \begin{macrocode}
\ExplSyntaxOff
%    \end{macrocode}
%
% \subsection{TikZ Pictures}
%
% New in TikZ3.0 is the ability to make pictures that can be reused.
% Penrose tiles seems an obvious use for this.
% These pictures can be placed alongside other tiles, matching by edge type.
%
% There are a variety of constants that are frequently used and reused, so we define them all here.
% These are the PGF versions.
%
%    \begin{macrocode}
\pgfmathsetmacro\pr@chphi{cos(18)}
\pgfmathsetmacro\pr@shphi{sin(18)}
\pgfmathsetmacro\pr@cphi{cos(36)}
\pgfmathsetmacro\pr@sphi{sin(36)}
\pgfmathsetmacro\pr@invphi{2/(sqrt(5)+1)}
\pgfmathsetmacro\pr@invphisq{\pr@invphi*\pr@invphi}
\pgfmathsetmacro\pr@ominvphisq{\pr@invphi - \pr@invphisq}
\pgfmathsetmacro\pr@ominvphi{1 - \pr@invphi}
%    \end{macrocode}
%
% The implementation is essentially the same for each, so only the first will be commented.
%    \begin{macrocode}
\tikzset{
%    \end{macrocode}
% The key \Verb+align with=<tile> along <edge>+ is used to set the parameters for placing a tile next to an existing one.
%    \begin{macrocode}
  align with/.code args={#1 along #2}{%
    \tikzset{
      Penrose/alignment location=#1,
      Penrose/alignment edge=#2,
    }%
  },
  Penrose/alignment location/.initial={},
  Penrose/alignment edge/.initial=a,
%    \end{macrocode}
% This is the code for setting up a \Verb+pic+.
%    \begin{macrocode}
  thin rhombus/.pic={
    \begin{scope}
%    \end{macrocode}
% Were we given a tile to align ourselves against?
%    \begin{macrocode}
    \pgfkeysgetvalue{/tikz/Penrose/alignment location}{\prloc}
    \ifx\prloc\pgfutil@empty
    \else
%    \end{macrocode}
% Yes, we were.
% So we adjust our position accordingly.
% The first job is to transform so that we're along the edge of the receiving tile.
%    \begin{macrocode}
    \begingroup
%    \end{macrocode}
% We get the locations of the start and end of the receiving tile.
% As \Verb+pic+ sets the node prefix, we have to temporarily suspend that (hence working in a group).
%    \begin{macrocode}
    \tikzset{name prefix ..}%
    \tikz@scan@one@point\pgfutil@firstofone(\prloc-edge \pgfkeysvalueof{/tikz/Penrose/alignment edge} start)%
    \global\pgf@xa=\pgf@x
    \global\pgf@ya=\pgf@y
    \tikz@scan@one@point\pgfutil@firstofone(\prloc-edge \pgfkeysvalueof{/tikz/Penrose/alignment edge} end)%
    \global\pgf@xb=\pgf@x
    \global\pgf@yb=\pgf@y
    \endgroup
%    \end{macrocode}
% We store the initial points in \Verb+\pgf@xa+ and \Verb+\pgf@ya+ but we want \Verb+\pgf@xb+ and \Verb+\pgf@yb+ to be a vector along the edge.
%    \begin{macrocode}
    \advance\pgf@xb by -\pgf@xa
    \advance\pgf@yb by -\pgf@ya
%    \end{macrocode}
% We shift to the start of the edge.
%    \begin{macrocode}
    \pgftransformshift{\pgfpoint{\pgf@xa}{\pgf@ya}}%
%    \end{macrocode}
% And normalise the vector along it.
%    \begin{macrocode}
    \pgfpointnormalised{\pgfpoint{\pgf@xb}{\pgf@yb}}
    \pgf@xb=\pgf@x
    \pgf@yb=\pgf@y
%    \end{macrocode}
% Now rotate so that the \(x\)--axis lies along the edge.
%    \begin{macrocode}
\pgftransformtriangle{\pgfpoint{0pt}{0pt}}{\pgfpoint{\pgf@xb}{\pgf@yb}}{\pgfpoint{-\pgf@yb}{\pgf@xb}}
%    \end{macrocode}
% The next job is to shift and rotate the current tile so that the correct edge ends up against the receiving tile.
%    \begin{macrocode}
    \if\pgfkeysvalueof{/tikz/Penrose/alignment edge}b\relax
    \pgftransformrotate{-18}%
    \pgftransformshift{\pgfpoint{-\pr@chphi cm}{\pr@shphi cm}}%
    \else
    \if\pgfkeysvalueof{/tikz/Penrose/alignment edge}B\relax
    \pgftransformrotate{18}%
    \else
    \if\pgfkeysvalueof{/tikz/Penrose/alignment edge}a\relax
    \pgftransformrotate{198}%
    \pgftransformshift{\pgfpoint{-2*\pr@chphi cm}{0 cm}}%
    \else
    \if\pgfkeysvalueof{/tikz/Penrose/alignment edge}A\relax
    \pgftransformrotate{162}%
    \pgftransformshift{\pgfpoint{-\pr@chphi cm}{-\pr@shphi cm}}%
    \fi\fi\fi\fi
    \fi
%    \end{macrocode}
% Now that the transformation is finalised, we can render the tile.
% We clip against the tile path so that the tiles don't ``bleed''.
% If we didn't do this, drawing the tile would result in overlaps which can look a bit ugly.
%    \begin{macrocode}
    \UsePenroseTile[clip]{thin rhombus}
    \UsePenroseTile[every Penrose tile/.try, every thin rhombus/.try, pic actions]{thin rhombus}
%    \end{macrocode}
% These draw the arcs that designate the joining rules.
% We draw full circles so that it doesn't matter what shape the tiles are.
%    \begin{macrocode}
\path[every circle arc/.try] (18:1) circle[radius=1/4];
\path[every long arc/.try] (-18:1) circle[radius=1/4];
%    \end{macrocode}
% Lastly, we put coordinates at each vertex, labelled by which edge they are.
%    \begin{macrocode}
\coordinate (-edge a start) at (0,0);
\coordinate (-edge a end) at (18:1);
\coordinate (-edge A start) at (18:1);
\coordinate (-edge A end) at (2*\pr@chphi,0);
\coordinate (-edge B start) at (2*\pr@chphi,0);
\coordinate (-edge B end) at (-18:1);
\coordinate (-edge b start) at (-18:1);
\coordinate (-edge b end) at (0,0);
    \end{scope}
  },
%    \end{macrocode}
% This is a shortcut for installing the \Verb+pic+ type.
%    \begin{macrocode}
  thin rhombus/.style={
    every Penrose pic/.try,
    pic type=thin rhombus,
  },
%    \end{macrocode}
% Same again, but for the thick rhombus.
%    \begin{macrocode}
  thick rhombus/.pic={
    \begin{scope}
    \pgfkeysgetvalue{/tikz/Penrose/alignment location}{\prloc}
    \ifx\prloc\pgfutil@empty
    \else
    \begingroup
    \tikzset{name prefix ..}%
    \tikz@scan@one@point\pgfutil@firstofone(\prloc-edge \pgfkeysvalueof{/tikz/Penrose/alignment edge} start)%
    \global\pgf@xa=\pgf@x
    \global\pgf@ya=\pgf@y
    \tikz@scan@one@point\pgfutil@firstofone(\prloc-edge \pgfkeysvalueof{/tikz/Penrose/alignment edge} end)%
    \global\pgf@xb=\pgf@x
    \global\pgf@yb=\pgf@y
    \endgroup
    \advance\pgf@xb by -\pgf@xa
    \advance\pgf@yb by -\pgf@ya
    \pgftransformshift{\pgfpoint{\pgf@xa}{\pgf@ya}}%
    \pgfpointnormalised{\pgfpoint{\pgf@xb}{\pgf@yb}}
    \pgf@xb=\pgf@x
    \pgf@yb=\pgf@y
\pgftransformtriangle{\pgfpoint{0pt}{0pt}}{\pgfpoint{\pgf@xb}{\pgf@yb}}{\pgfpoint{-\pgf@yb}{\pgf@xb}}
    \if\pgfkeysvalueof{/tikz/Penrose/alignment edge}b\relax
    \pgftransformrotate{144}%
    \pgftransformshift{\pgfpoint{-\pr@cphi cm}{-\pr@sphi cm}}%
    \else
    \if\pgfkeysvalueof{/tikz/Penrose/alignment edge}B\relax
    \pgftransformrotate{36}%
    \else
    \if\pgfkeysvalueof{/tikz/Penrose/alignment edge}a\relax
    \pgftransformrotate{-36}%
    \pgftransformshift{\pgfpoint{-\pr@cphi cm}{\pr@sphi cm}}%
    \else
    \if\pgfkeysvalueof{/tikz/Penrose/alignment edge}A\relax
    \pgftransformrotate{216}%
    \pgftransformshift{\pgfpoint{-2*\pr@cphi cm}{0 cm}}%
    \fi\fi\fi\fi
    \fi
    \UsePenroseTile[clip]{thick rhombus}
    \UsePenroseTile[every Penrose tile/.try, every thick rhombus/.try, pic actions]{thick rhombus}
\path[every circle arc/.try] (2*\pr@cphi,0) circle[radius=1/4];
\path[every long arc/.try] (0,0) circle[radius=3/4];
\coordinate (-edge B start) at (0,0);
\coordinate (-edge B end) at (36:1);
\coordinate (-edge a start) at (36:1);
\coordinate (-edge a end) at (2*\pr@cphi,0);
\coordinate (-edge A start) at (2*\pr@cphi,0);
\coordinate (-edge A end) at (-36:1);
\coordinate (-edge b start) at (-36:1);
\coordinate (-edge b end) at (0,0);
    \end{scope}
  },
  thick rhombus/.style={
    every Penrose pic/.try,
    pic type=thick rhombus,
  },
%    \end{macrocode}
% Now the kite.
%    \begin{macrocode}
  kite/.pic={
    \begin{scope}
    \pgfkeysgetvalue{/tikz/Penrose/alignment location}{\prloc}
    \ifx\prloc\pgfutil@empty
    \else
    \begingroup
    \tikzset{name prefix ..}%
    \tikz@scan@one@point\pgfutil@firstofone(\prloc-edge \pgfkeysvalueof{/tikz/Penrose/alignment edge} start)%
    \global\pgf@xa=\pgf@x
    \global\pgf@ya=\pgf@y
    \tikz@scan@one@point\pgfutil@firstofone(\prloc-edge \pgfkeysvalueof{/tikz/Penrose/alignment edge} end)%
    \global\pgf@xb=\pgf@x
    \global\pgf@yb=\pgf@y
    \endgroup
    \advance\pgf@xb by -\pgf@xa
    \advance\pgf@yb by -\pgf@ya
    \pgftransformshift{\pgfpoint{\pgf@xa}{\pgf@ya}}%
    \pgfpointnormalised{\pgfpoint{\pgf@xb}{\pgf@yb}}
    \pgf@xb=\pgf@x
    \pgf@yb=\pgf@y
\pgftransformtriangle{\pgfpoint{0pt}{0pt}}{\pgfpoint{\pgf@xb}{\pgf@yb}}{\pgfpoint{-\pgf@yb}{\pgf@xb}}
    \if\pgfkeysvalueof{/tikz/Penrose/alignment edge}c\relax
    \pgftransformrotate{-72}%
    \pgftransformshift{\pgfpoint{-\pr@cphi cm}{\pr@sphi cm}}%
    \else
    \if\pgfkeysvalueof{/tikz/Penrose/alignment edge}C\relax
    \pgftransformrotate{-108}%
    \pgftransformshift{\pgfpoint{-1 cm}{0 cm}}%
    \else
    \if\pgfkeysvalueof{/tikz/Penrose/alignment edge}a\relax
    \pgftransformrotate{36}%
    \else
    \if\pgfkeysvalueof{/tikz/Penrose/alignment edge}A\relax
    \pgftransformrotate{144}%
    \pgftransformshift{\pgfpoint{-\pr@cphi cm}{-\pr@sphi cm}}%
    \fi\fi\fi\fi
    \fi
    \UsePenroseTile[clip]{kite}
    \UsePenroseTile[every Penrose tile/.try, every kite/.try, pic actions]{kite}
\path[every circle arc/.try] (0,0) circle[radius=\pr@invphi];
\path[every long arc/.try] (1,0) circle[radius=\pr@invphisq];
\coordinate (-edge a start) at (0,0);
\coordinate (-edge a end) at (36:1);
\coordinate (-edge c start) at (36:1);
\coordinate (-edge c end) at (1,0);
\coordinate (-edge C start) at (1,0);
\coordinate (-edge C end) at (-36:1);
\coordinate (-edge A start) at (-36:1);
\coordinate (-edge A end) at (0,0);
    \end{scope}
  },
%    \end{macrocode}
% The dart is next.
%    \begin{macrocode}
  dart/.pic={
    \begin{scope}
    \pgfkeysgetvalue{/tikz/Penrose/alignment location}{\prloc}
    \ifx\prloc\pgfutil@empty
    \else
    \begingroup
    \tikzset{name prefix ..}%
    \tikz@scan@one@point\pgfutil@firstofone(\prloc-edge \pgfkeysvalueof{/tikz/Penrose/alignment edge} start)%
    \global\pgf@xa=\pgf@x
    \global\pgf@ya=\pgf@y
    \tikz@scan@one@point\pgfutil@firstofone(\prloc-edge \pgfkeysvalueof{/tikz/Penrose/alignment edge} end)%
    \global\pgf@xb=\pgf@x
    \global\pgf@yb=\pgf@y
    \endgroup
    \advance\pgf@xb by -\pgf@xa
    \advance\pgf@yb by -\pgf@ya
    \pgftransformshift{\pgfpoint{\pgf@xa}{\pgf@ya}}%
    \pgfpointnormalised{\pgfpoint{\pgf@xb}{\pgf@yb}}
    \pgf@xb=\pgf@x
    \pgf@yb=\pgf@y
\pgftransformtriangle{\pgfpoint{0pt}{0pt}}{\pgfpoint{\pgf@xb}{\pgf@yb}}{\pgfpoint{-\pgf@yb}{\pgf@xb}}
    \if\pgfkeysvalueof{/tikz/Penrose/alignment edge}c\relax
    \pgftransformrotate{108}%
    \else
    \if\pgfkeysvalueof{/tikz/Penrose/alignment edge}C\relax
    \pgftransformrotate{72}%
    \pgftransformshift{\pgfpoint{\pr@invphi*\pr@shphi cm}{-\pr@invphi*\pr@chphi cm}}%
    \else
    \if\pgfkeysvalueof{/tikz/Penrose/alignment edge}a\relax
    \pgftransformrotate{-36}%
    \pgftransformshift{\pgfpoint{\pr@invphi*\pr@shphi cm}{\pr@invphi*\pr@chphi cm}}%
    \else
    \if\pgfkeysvalueof{/tikz/Penrose/alignment edge}A\relax
    \pgftransformrotate{216}%
    \pgftransformshift{\pgfpoint{-\pr@invphi cm}{0 cm}}%
    \fi\fi\fi\fi
    \fi
    \UsePenroseTile[clip]{dart}
    \UsePenroseTile[every Penrose tile/.try, every dart/.try, pic actions]{dart}
\path[every circle arc/.try] (\pr@invphi,0) circle[radius=\pr@ominvphi];
\path[every long arc/.try] (0,0) circle[radius=\pr@ominvphisq];
\coordinate (-edge c start) at (0,0);
\coordinate (-edge c end) at (108:\pr@invphi);
\coordinate (-edge a start) at (108:\pr@invphi);
\coordinate (-edge a end) at (\pr@invphi,0);
\coordinate (-edge A start) at (\pr@invphi,0);
\coordinate (-edge A end) at (-108:\pr@invphi);
\coordinate (-edge C start) at (-108:\pr@invphi);
\coordinate (-edge C end) at (0,0);
    \end{scope}
  },
  kite/.style={
    every Penrose pic/.try,
    pic type=kite,
  },
  dart/.style={
    every Penrose pic/.try,
    pic type=dart,
  },
%    \end{macrocode}
% The golden triangle.
%    \begin{macrocode}
  golden triangle/.pic={
    \begin{scope}
    \pgfkeysgetvalue{/tikz/Penrose/alignment location}{\prloc}
    \ifx\prloc\pgfutil@empty
    \else
    \begingroup
    \tikzset{name prefix ..}%
    \tikz@scan@one@point\pgfutil@firstofone(\prloc-edge \pgfkeysvalueof{/tikz/Penrose/alignment edge} start)%
    \global\pgf@xa=\pgf@x
    \global\pgf@ya=\pgf@y
    \tikz@scan@one@point\pgfutil@firstofone(\prloc-edge \pgfkeysvalueof{/tikz/Penrose/alignment edge} end)%
    \global\pgf@xb=\pgf@x
    \global\pgf@yb=\pgf@y
    \endgroup
    \advance\pgf@xb by -\pgf@xa
    \advance\pgf@yb by -\pgf@ya
    \pgftransformshift{\pgfpoint{\pgf@xa}{\pgf@ya}}%
    \pgfpointnormalised{\pgfpoint{\pgf@xb}{\pgf@yb}}
    \pgf@xb=\pgf@x
    \pgf@yb=\pgf@y
\pgftransformtriangle{\pgfpoint{0pt}{0pt}}{\pgfpoint{\pgf@xb}{\pgf@yb}}{\pgfpoint{-\pgf@yb}{\pgf@xb}}
    \if\pgfkeysvalueof{/tikz/Penrose/alignment edge}B\relax
    \pgftransformrotate{18}%
    \else
    \if\pgfkeysvalueof{/tikz/Penrose/alignment edge}C\relax
    \pgftransformrotate{-90}%
    \pgftransformshift{\pgfpoint{-\pr@chphi cm}{\pr@shphi cm}}%
    \else
    \if\pgfkeysvalueof{/tikz/Penrose/alignment edge}A\relax
    \pgftransformrotate{162}%
    \pgftransformshift{\pgfpoint{-\pr@chphi cm}{-\pr@shphi cm}}%
    \fi\fi\fi
    \fi
    \UsePenroseTile[clip]{golden triangle}
    \UsePenroseTile[every Penrose tile/.try, every golden triangle/.try, pic actions]{golden triangle}
\coordinate (-edge a start) at (0,0);
\coordinate (-edge a end) at (18:1);
\coordinate (-edge c start) at (18:1);
\coordinate (-edge c end) at (-18:1);
\coordinate (-edge b start) at (-18:1);
\coordinate (-edge b end) at (0,0);
    \end{scope}
  },
  golden triangle/.style={
    every Penrose pic/.try,
    pic type=golden triangle,
  },
%    \end{macrocode}
% The reverse golden triangle (is there a better name?).
%    \begin{macrocode}
  reverse golden triangle/.pic={
    \begin{scope}
    \pgfkeysgetvalue{/tikz/Penrose/alignment location}{\prloc}
    \ifx\prloc\pgfutil@empty
    \else
    \begingroup
    \tikzset{name prefix ..}%
    \tikz@scan@one@point\pgfutil@firstofone(\prloc-edge \pgfkeysvalueof{/tikz/Penrose/alignment edge} start)%
    \global\pgf@xa=\pgf@x
    \global\pgf@ya=\pgf@y
    \tikz@scan@one@point\pgfutil@firstofone(\prloc-edge \pgfkeysvalueof{/tikz/Penrose/alignment edge} end)%
    \global\pgf@xb=\pgf@x
    \global\pgf@yb=\pgf@y
    \endgroup
    \advance\pgf@xb by -\pgf@xa
    \advance\pgf@yb by -\pgf@ya
    \pgftransformshift{\pgfpoint{\pgf@xa}{\pgf@ya}}%
    \pgfpointnormalised{\pgfpoint{\pgf@xb}{\pgf@yb}}
    \pgf@xb=\pgf@x
    \pgf@yb=\pgf@y
\pgftransformtriangle{\pgfpoint{0pt}{0pt}}{\pgfpoint{\pgf@xb}{\pgf@yb}}{\pgfpoint{-\pgf@yb}{\pgf@xb}}
    \if\pgfkeysvalueof{/tikz/Penrose/alignment edge}b\relax
    \pgftransformrotate{162}%
    \pgftransformshift{\pgfpoint{-\pr@chphi cm}{-\pr@shphi cm}}%
    \else
    \if\pgfkeysvalueof{/tikz/Penrose/alignment edge}c\relax
    \pgftransformrotate{-90}%
    \pgftransformshift{\pgfpoint{-\pr@chphi cm}{\pr@shphi cm}}%
    \else
    \if\pgfkeysvalueof{/tikz/Penrose/alignment edge}a\relax
    \pgftransformrotate{18}%
    \fi\fi\fi
    \fi
    \UsePenroseTile[clip]{reverse golden triangle}
    \UsePenroseTile[every Penrose tile/.try, every reverse golden triangle/.try, pic actions]{reverse golden triangle}
\coordinate (-edge B start) at (0,0);
\coordinate (-edge B end) at (18:1);
\coordinate (-edge C start) at (18:1);
\coordinate (-edge C end) at (-18:1);
\coordinate (-edge A start) at (-18:1);
\coordinate (-edge A end) at (0,0);
    \end{scope}
  },
  reverse golden triangle/.style={
    every Penrose pic/.try,
    pic type=reverse golden triangle,
  },
%    \end{macrocode}
% The golden gnomon.
%    \begin{macrocode}
  golden gnomon/.pic={
    \begin{scope}
    \pgfkeysgetvalue{/tikz/Penrose/alignment location}{\prloc}
    \ifx\prloc\pgfutil@empty
    \else
    \begingroup
    \tikzset{name prefix ..}%
    \tikz@scan@one@point\pgfutil@firstofone(\prloc-edge \pgfkeysvalueof{/tikz/Penrose/alignment edge} start)%
    \global\pgf@xa=\pgf@x
    \global\pgf@ya=\pgf@y
    \tikz@scan@one@point\pgfutil@firstofone(\prloc-edge \pgfkeysvalueof{/tikz/Penrose/alignment edge} end)%
    \global\pgf@xb=\pgf@x
    \global\pgf@yb=\pgf@y
    \endgroup
    \advance\pgf@xb by -\pgf@xa
    \advance\pgf@yb by -\pgf@ya
    \pgftransformshift{\pgfpoint{\pgf@xa}{\pgf@ya}}%
    \pgfpointnormalised{\pgfpoint{\pgf@xb}{\pgf@yb}}
    \pgf@xb=\pgf@x
    \pgf@yb=\pgf@y
\pgftransformtriangle{\pgfpoint{0pt}{0pt}}{\pgfpoint{\pgf@xb}{\pgf@yb}}{\pgfpoint{-\pgf@yb}{\pgf@xb}}
    \if\pgfkeysvalueof{/tikz/Penrose/alignment edge}c\relax
    \pgftransformrotate{144}%
    \pgftransformshift{\pgfpoint{-\pr@cphi cm}{-\pr@sphi cm}}%
    \else
    \if\pgfkeysvalueof{/tikz/Penrose/alignment edge}B\relax
    \pgftransformrotate{-144}%
    \pgftransformshift{\pgfpoint{-2*\pr@cphi cm}{0 cm}}%
    \else
    \if\pgfkeysvalueof{/tikz/Penrose/alignment edge}a\relax
    \fi\fi\fi
    \fi
    \UsePenroseTile[clip]{golden gnomon}
    \UsePenroseTile[every Penrose tile/.try, every golden gnomon/.try, pic actions]{golden gnomon}
\coordinate (-edge C start) at (0,0);
\coordinate (-edge C end) at (36:1);
\coordinate (-edge b start) at (36:1);
\coordinate (-edge b end) at (2*\pr@cphi,0);
\coordinate (-edge A start) at (2*\pr@cphi,0);
\coordinate (-edge A end) at (0,0);
    \end{scope}
  },
  golden gnomon/.style={
    every Penrose pic/.try,
    pic type=golden gnomon,
  },
%    \end{macrocode}
% The reverse golden gnomon.
%    \begin{macrocode}
  reverse golden gnomon/.pic={
    \begin{scope}
    \pgfkeysgetvalue{/tikz/Penrose/alignment location}{\prloc}
    \ifx\prloc\pgfutil@empty
    \else
    \begingroup
    \tikzset{name prefix ..}%
    \tikz@scan@one@point\pgfutil@firstofone(\prloc-edge \pgfkeysvalueof{/tikz/Penrose/alignment edge} start)%
    \global\pgf@xa=\pgf@x
    \global\pgf@ya=\pgf@y
    \tikz@scan@one@point\pgfutil@firstofone(\prloc-edge \pgfkeysvalueof{/tikz/Penrose/alignment edge} end)%
    \global\pgf@xb=\pgf@x
    \global\pgf@yb=\pgf@y
    \endgroup
    \advance\pgf@xb by -\pgf@xa
    \advance\pgf@yb by -\pgf@ya
    \pgftransformshift{\pgfpoint{\pgf@xa}{\pgf@ya}}%
    \pgfpointnormalised{\pgfpoint{\pgf@xb}{\pgf@yb}}
    \pgf@xb=\pgf@x
    \pgf@yb=\pgf@y
\pgftransformtriangle{\pgfpoint{0pt}{0pt}}{\pgfpoint{\pgf@xb}{\pgf@yb}}{\pgfpoint{-\pgf@yb}{\pgf@xb}}
    \if\pgfkeysvalueof{/tikz/Penrose/alignment edge}C\relax
    \pgftransformrotate{36}%
    \else
    \if\pgfkeysvalueof{/tikz/Penrose/alignment edge}b\relax
    \pgftransformrotate{-36}%
    \pgftransformshift{\pgfpoint{-\pr@cphi cm}{\pr@sphi cm}}%
    \else
    \if\pgfkeysvalueof{/tikz/Penrose/alignment edge}A\relax
    \pgftransformrotate{180}%
    \pgftransformshift{\pgfpoint{-2*\pr@cphi cm}{0 cm}}%
    \fi\fi\fi
    \fi
    \UsePenroseTile[clip]{reverse golden gnomon}
    \UsePenroseTile[every Penrose tile/.try, every reverse golden gnomon/.try, pic actions]{reverse golden gnomon}
\coordinate (-edge a start) at (0,0);
\coordinate (-edge a end) at (2*\pr@cphi,0);
\coordinate (-edge B start) at (2*\pr@cphi,0);
\coordinate (-edge B end) at (-36:1);
\coordinate (-edge c start) at (-36:1);
\coordinate (-edge c end) at (0,0);
    \end{scope}
  },
  reverse golden gnomon/.style={
    every Penrose pic/.try,
    pic type=reverse golden gnomon,
  },
}
%    \end{macrocode}
%
% \iffalse
%</library>
% \fi
%\Finale
\endinput
